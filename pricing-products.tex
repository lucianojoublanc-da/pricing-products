%        File: pricing-products.tex
%     Created: Mon Jul 05 11:00 am 2021 C
% Last Change: Mon Jul 05 11:00 am 2021 C
%
\documentclass[a4paper]{article}
\usepackage[]{amsmath, amssymb}
\begin{document}

\section*{Exercise}
For the following products (in the given order! - you can skip over the starred * ones):

\begin{enumerate}
\item a Margrabe option on Apple and Amazon stock.
\item a (interest-rate) caplet. This should reduce to the Black-76 formula, which is a special case of the Margrabe formula.
\item a zero coupon bond, with the Vasicek model as the risk-free rate. If you get stuck, you can look at Jamshidian 1989 ``An Exact Bond Option Formula''. I think the challenge here is finding the equivalent martingale measure, in order to solve analytically.
\item a zero coupon bond, with the Cox-Ingersoll-Ross model as the risk-free rate. Compare to your previous result. Note this has a different distribution.
\item fixed rate, coupon paying bond using a short-rate model of your choice (pick a simple one - you'll need to calibrate it to market data!). Pick a bond with a short time-to-maturity (ideally $T-t = 2$), to simplify your calculations for numerical results.
\item * floating rate, coupon paying bond using the same short-rate model from above. Pick a bond with a short time-to-maturity, to simplify your calculations for numerical results.
\item the same fixed rate bond from above, using Brace-Gatarek-Musiela aka ``LIBOR forward-rate market'' model. This will require a change of measure.
\item the same floating rate bond from above, using Brace-Gatarek-Musiela aka ``LIBOR forward-rate market'' model. This will require a change of measure. Show your result is consistent with market cap prices (you derived this in (2)). See (Brigo, Mercurio proposition 6.4.1 - Equivalence between LFM and Black's caplet prices).
\item interest rate swap pay fixed / receive floating (Think about how to model this from a cash-flow perspective).
\item * a European call option, with stock price as GBM, and a stochastic short-rate model of your choice. If it's not possible to price this using a Martingale approach, explain why.
\item * a European call option, with stock price as GBM, and a stochastic volatility model of your choice. If it's not possible to price this using a Martingale approach, explain why.
\end{enumerate}

Perform the following steps, showing all your working:

\section{Write down the Eber/Jones expression}

Use this definition:

\begin{align*}
Claim := & Zero \\
|& One \\
|& Give \; Claim \\
|& Scale \; Observable \; Claim \\
|& And \; Claim \; Claim \\
|& Or \; Claim \; Claim \\
|& Cond \; Inequality \; Claim \; Claim \\
|& When \; Inequality \; Cliam \\
|& Anytime \; Inequality \; Claim \\
|& Until \; Inequality \; Claim \\
Inequality := & TimeGte \; t \\
| & Lte \; Observable \; Observable \\
\end{align*}
I've included the below for completeness, but you can ignore this and just write down $Observe$ as $S_t$ and $Const$ as e.g. $K$, and the operators as $+, -, \times$.
\begin{align*}
Observable := & Const \\
| & Observe \\
| & Add \; Observable \; Observable \\
| & Neg \; Observable \\
| & Mul \; Observable \; Observable
\end{align*}


For example, for a European put option:

\begin{align*}
  t &:= \text{ time to maturity} \\
  K &:= \text{ strike price} \\
  S_t &:= \text{ the (observed) stock price, follows GBM} \\
  Q_t & := \text{ the USD discount rate } \\
  & When (TimeGte \; T) (Or (Scale (K - S_t) (One Q_t)) Zero)
\end{align*}


Assume we are pricing in USD.

Check that this matches with what is produced by our implementation in https://github.com/digital-asset/contingent-claims

\section{Transform to expectations using E1-E10}

For example, the above would translate (co)recursively to

\begin{align*}
& When \quad (TimeGte \; T) \quad (Or \; (Scale \: (K - S_t) \: (One Q_t)) \; Zero) \\
= & Q_t \mathbb{E}^Q[ \frac{ Or \; (Scale \: (K - S_{t = T}) \: (One Q_{t = T})) \; Zero } { Q_T } | \mathcal{F}_t ] & \text{ by E8} \\
= & Q_t \mathbb{E}^Q[ \frac{ max ( Scale \: (K - S_T) \: (One Q_T) , Zero ) } { Q_T } | \mathcal{F}_t ] & \text{ by E6} \\
& \text{ by definition of $max$, and assuming no correlation to $Zero$, } \\
= & Q_t \mathbb{E}^Q[ \frac{  A \times I_A  + Zero \times I_{A^c}} { Q_T } | \mathcal{F}_t ] , & \{ A : Scale \: (K - S_T) \: (One Q_T) \geq Zero \} \\
= & Q_t \mathbb{E}^Q[ \frac{  A \times I_A } { Q_T } | \mathcal{F}_t ] , & \{ A : (K - S_T) \times 1 \geq 0 \} \text{ by E1, E2, E4} \\
= & Q_t \mathbb{E}^Q[ \frac{  (K - S_T) I_{K \geq S_T} } { Q_T } | \mathcal{F}_t ] \\
\end{align*}

Note that here I've \emph{not} written everything in terms of filtered expectations. I'm not sure anymore if this is helpful - I think that it's enough to observe that the tower property applies to nested $When$ expressions; also, you established that $Until$ in it's current form is not an expectation. 

Also note that I've now expaned this co-recursively (i.e. from the outside in / top to bottom) rather than recursively (inside-out / leaves-to-root). This is so the assignment of $T$ is clear.

\section{Simplify the expectation to an analytical price}

This is where you would use 'change of numeraire' if necessary - in this example it's not.

\begin{align*}
& Q_t \mathbb{E}^Q[ \frac{  (K - S_T) I_{K \geq S_T} } { Q_T } | \mathcal{F}_t ] \\
= & Q_t \mathbb{E}^Q[ \frac{K} {Q_T} I_{K \geq S_T} | \mathcal{F}_t ] - Q_t \mathbb{E}^Q[ \frac{S_T} {Q_T} I_{K \geq S_T} | \mathcal{F}_t ] \\
& \text{quotient in the RHS is a Radom-Nykodim derivative, so by FAPF $\frac{S_t}{Q_t} = E^Q[\frac{S_T}{Q_T} | F_t]$} \\
& \text{and for LHS term, let's assume $Q(t)$ is deterministic, then} \\
= & \frac{Q^t} {Q_T} K \mathbb{E}^Q[  I_{K \geq S_T} | \mathcal{F}_t ] - \frac{Q_t}{Q_t} S_t \mathbb{E}^{*}[  I_{K \geq S^*_T} | \mathcal{F}_t ] \\
= & K e^{-r(T-t)} \mathbb{P}^Q[K \geq S_T] - S_t \mathbb{P}^{*}[K \geq S^*_T] \\
\end{align*}

With some more working, and taking care to use the appropriate measure $Q$ or $*$, this reduces to the Black-Scholes-Merton formula.

\section{Verify your results by numerical approximation (or otherwise)}

You will need market data for this, e.g. from a Bloomberg terminal or website. Share this data so the results are reproducible.

Do your modelled price match the actual market prices? If not, why not?

Be clear about what data you have used to calibrate your model parameters.

\end{document}
